% Options for packages loaded elsewhere
\PassOptionsToPackage{unicode}{hyperref}
\PassOptionsToPackage{hyphens}{url}
%
\documentclass[
]{article}
\usepackage{lmodern}
\usepackage{amssymb,amsmath}
\usepackage{ifxetex,ifluatex}
\ifnum 0\ifxetex 1\fi\ifluatex 1\fi=0 % if pdftex
  \usepackage[T1]{fontenc}
  \usepackage[utf8]{inputenc}
  \usepackage{textcomp} % provide euro and other symbols
\else % if luatex or xetex
  \usepackage{unicode-math}
  \defaultfontfeatures{Scale=MatchLowercase}
  \defaultfontfeatures[\rmfamily]{Ligatures=TeX,Scale=1}
\fi
% Use upquote if available, for straight quotes in verbatim environments
\IfFileExists{upquote.sty}{\usepackage{upquote}}{}
\IfFileExists{microtype.sty}{% use microtype if available
  \usepackage[]{microtype}
  \UseMicrotypeSet[protrusion]{basicmath} % disable protrusion for tt fonts
}{}
\makeatletter
\@ifundefined{KOMAClassName}{% if non-KOMA class
  \IfFileExists{parskip.sty}{%
    \usepackage{parskip}
  }{% else
    \setlength{\parindent}{0pt}
    \setlength{\parskip}{6pt plus 2pt minus 1pt}}
}{% if KOMA class
  \KOMAoptions{parskip=half}}
\makeatother
\usepackage{xcolor}
\IfFileExists{xurl.sty}{\usepackage{xurl}}{} % add URL line breaks if available
\IfFileExists{bookmark.sty}{\usepackage{bookmark}}{\usepackage{hyperref}}
\hypersetup{
  pdftitle={final-story},
  hidelinks,
  pdfcreator={LaTeX via pandoc}}
\urlstyle{same} % disable monospaced font for URLs
\usepackage[margin=1in]{geometry}
\usepackage{color}
\usepackage{fancyvrb}
\newcommand{\VerbBar}{|}
\newcommand{\VERB}{\Verb[commandchars=\\\{\}]}
\DefineVerbatimEnvironment{Highlighting}{Verbatim}{commandchars=\\\{\}}
% Add ',fontsize=\small' for more characters per line
\usepackage{framed}
\definecolor{shadecolor}{RGB}{248,248,248}
\newenvironment{Shaded}{\begin{snugshade}}{\end{snugshade}}
\newcommand{\AlertTok}[1]{\textcolor[rgb]{0.94,0.16,0.16}{#1}}
\newcommand{\AnnotationTok}[1]{\textcolor[rgb]{0.56,0.35,0.01}{\textbf{\textit{#1}}}}
\newcommand{\AttributeTok}[1]{\textcolor[rgb]{0.77,0.63,0.00}{#1}}
\newcommand{\BaseNTok}[1]{\textcolor[rgb]{0.00,0.00,0.81}{#1}}
\newcommand{\BuiltInTok}[1]{#1}
\newcommand{\CharTok}[1]{\textcolor[rgb]{0.31,0.60,0.02}{#1}}
\newcommand{\CommentTok}[1]{\textcolor[rgb]{0.56,0.35,0.01}{\textit{#1}}}
\newcommand{\CommentVarTok}[1]{\textcolor[rgb]{0.56,0.35,0.01}{\textbf{\textit{#1}}}}
\newcommand{\ConstantTok}[1]{\textcolor[rgb]{0.00,0.00,0.00}{#1}}
\newcommand{\ControlFlowTok}[1]{\textcolor[rgb]{0.13,0.29,0.53}{\textbf{#1}}}
\newcommand{\DataTypeTok}[1]{\textcolor[rgb]{0.13,0.29,0.53}{#1}}
\newcommand{\DecValTok}[1]{\textcolor[rgb]{0.00,0.00,0.81}{#1}}
\newcommand{\DocumentationTok}[1]{\textcolor[rgb]{0.56,0.35,0.01}{\textbf{\textit{#1}}}}
\newcommand{\ErrorTok}[1]{\textcolor[rgb]{0.64,0.00,0.00}{\textbf{#1}}}
\newcommand{\ExtensionTok}[1]{#1}
\newcommand{\FloatTok}[1]{\textcolor[rgb]{0.00,0.00,0.81}{#1}}
\newcommand{\FunctionTok}[1]{\textcolor[rgb]{0.00,0.00,0.00}{#1}}
\newcommand{\ImportTok}[1]{#1}
\newcommand{\InformationTok}[1]{\textcolor[rgb]{0.56,0.35,0.01}{\textbf{\textit{#1}}}}
\newcommand{\KeywordTok}[1]{\textcolor[rgb]{0.13,0.29,0.53}{\textbf{#1}}}
\newcommand{\NormalTok}[1]{#1}
\newcommand{\OperatorTok}[1]{\textcolor[rgb]{0.81,0.36,0.00}{\textbf{#1}}}
\newcommand{\OtherTok}[1]{\textcolor[rgb]{0.56,0.35,0.01}{#1}}
\newcommand{\PreprocessorTok}[1]{\textcolor[rgb]{0.56,0.35,0.01}{\textit{#1}}}
\newcommand{\RegionMarkerTok}[1]{#1}
\newcommand{\SpecialCharTok}[1]{\textcolor[rgb]{0.00,0.00,0.00}{#1}}
\newcommand{\SpecialStringTok}[1]{\textcolor[rgb]{0.31,0.60,0.02}{#1}}
\newcommand{\StringTok}[1]{\textcolor[rgb]{0.31,0.60,0.02}{#1}}
\newcommand{\VariableTok}[1]{\textcolor[rgb]{0.00,0.00,0.00}{#1}}
\newcommand{\VerbatimStringTok}[1]{\textcolor[rgb]{0.31,0.60,0.02}{#1}}
\newcommand{\WarningTok}[1]{\textcolor[rgb]{0.56,0.35,0.01}{\textbf{\textit{#1}}}}
\usepackage{graphicx,grffile}
\makeatletter
\def\maxwidth{\ifdim\Gin@nat@width>\linewidth\linewidth\else\Gin@nat@width\fi}
\def\maxheight{\ifdim\Gin@nat@height>\textheight\textheight\else\Gin@nat@height\fi}
\makeatother
% Scale images if necessary, so that they will not overflow the page
% margins by default, and it is still possible to overwrite the defaults
% using explicit options in \includegraphics[width, height, ...]{}
\setkeys{Gin}{width=\maxwidth,height=\maxheight,keepaspectratio}
% Set default figure placement to htbp
\makeatletter
\def\fps@figure{htbp}
\makeatother
\setlength{\emergencystretch}{3em} % prevent overfull lines
\providecommand{\tightlist}{%
  \setlength{\itemsep}{0pt}\setlength{\parskip}{0pt}}
\setcounter{secnumdepth}{-\maxdimen} % remove section numbering

\title{final-story}
\author{}
\date{\vspace{-2.5em}}

\begin{document}
\maketitle

\hypertarget{load-the-janitor-and-tidyverse-packages}{%
\subsubsection{Load the janitor and tidyverse
packages}\label{load-the-janitor-and-tidyverse-packages}}

\begin{Shaded}
\begin{Highlighting}[]
\CommentTok{#install.packages("janitor")}
\KeywordTok{library}\NormalTok{(janitor)}
\end{Highlighting}
\end{Shaded}

\begin{verbatim}
## 
## Attaching package: 'janitor'
\end{verbatim}

\begin{verbatim}
## The following objects are masked from 'package:stats':
## 
##     chisq.test, fisher.test
\end{verbatim}

\begin{Shaded}
\begin{Highlighting}[]
\CommentTok{#install.packages("tidyverse")}
\KeywordTok{library}\NormalTok{(tidyverse)}
\end{Highlighting}
\end{Shaded}

\begin{verbatim}
## -- Attaching packages ---------------------------------------------------------------------------------------------------- tidyverse 1.3.0 --
\end{verbatim}

\begin{verbatim}
## v ggplot2 3.3.0     v purrr   0.3.4
## v tibble  3.0.0     v dplyr   0.8.5
## v tidyr   1.0.2     v stringr 1.4.0
## v readr   1.3.1     v forcats 0.5.0
\end{verbatim}

\begin{verbatim}
## -- Conflicts ------------------------------------------------------------------------------------------------------- tidyverse_conflicts() --
## x dplyr::filter() masks stats::filter()
## x dplyr::lag()    masks stats::lag()
\end{verbatim}

\begin{Shaded}
\begin{Highlighting}[]
\KeywordTok{library}\NormalTok{(lubridate)}
\end{Highlighting}
\end{Shaded}

\begin{verbatim}
## 
## Attaching package: 'lubridate'
\end{verbatim}

\begin{verbatim}
## The following objects are masked from 'package:dplyr':
## 
##     intersect, setdiff, union
\end{verbatim}

\begin{verbatim}
## The following objects are masked from 'package:base':
## 
##     date, intersect, setdiff, union
\end{verbatim}

\hypertarget{reporting-question}{%
\subsubsection{Reporting Question}\label{reporting-question}}

\hypertarget{i-noticed-that-a-large-amount-of-donations-came-from-kim-coley-herself-as-in-kind-donations.-im-curious-about-how-well-campaigns-turn-out-depending-on-how-much-the-candidate-invests-in-themselves.-is-there-a-correlation-between-how-much-money-a-candidate-invests-in-their-own-campaign-and-how-much-money-others-invest-in-their-campaign-do-candidates-who-invest-a-certain-amount-into-their-campaign-have-a-higher-chance-of-winning}{%
\paragraph{I noticed that a large amount of donations came from Kim
Coley herself as in-kind donations. I'm curious about how well campaigns
turn out depending on how much the candidate invests in themselves. Is
there a correlation between how much money a candidate invests in their
own campaign and how much money others invest in their campaign? Do
candidates who invest a certain amount into their campaign have a higher
chance of
winning?}\label{i-noticed-that-a-large-amount-of-donations-came-from-kim-coley-herself-as-in-kind-donations.-im-curious-about-how-well-campaigns-turn-out-depending-on-how-much-the-candidate-invests-in-themselves.-is-there-a-correlation-between-how-much-money-a-candidate-invests-in-their-own-campaign-and-how-much-money-others-invest-in-their-campaign-do-candidates-who-invest-a-certain-amount-into-their-campaign-have-a-higher-chance-of-winning}}

\hypertarget{get-campaign-finance-data}{%
\subsubsection{Get campaign finance
data}\label{get-campaign-finance-data}}

\hypertarget{this-data-includes-all-2018-quarterly-reports-for-2018-nc-senate-candidates-where-available}{%
\paragraph{This data includes all 2018 quarterly reports for 2018 NC
Senate candidates where
available}\label{this-data-includes-all-2018-quarterly-reports-for-2018-nc-senate-candidates-where-available}}

\begin{Shaded}
\begin{Highlighting}[]
\CommentTok{# Reads in candidate data}
\NormalTok{candidateData <-}\StringTok{ }\KeywordTok{read_csv}\NormalTok{(}\StringTok{"../python/senate2018/urls.csv"}\NormalTok{)}
\end{Highlighting}
\end{Shaded}

\begin{verbatim}
## Parsed with column specification:
## cols(
##   district = col_double(),
##   win = col_double(),
##   margin = col_double(),
##   firstname = col_character(),
##   aliases = col_character(),
##   lastname = col_character(),
##   url = col_character()
## )
\end{verbatim}

\begin{Shaded}
\begin{Highlighting}[]
\NormalTok{receiptsAll <-}\StringTok{ }\KeywordTok{list}\NormalTok{()}

\CommentTok{# Given a district number and last name, consolidates the quarterly receipts for a candidate}
\CommentTok{# Returns a dataframe of receipts from 2018}
\NormalTok{consolidateReceipts <-}\StringTok{ }\ControlFlowTok{function}\NormalTok{(district, lastname) \{}
  \CommentTok{# Creates an empty dataframe with column names}
\NormalTok{  rcpts_complete <-}\StringTok{ }\KeywordTok{data.frame}\NormalTok{(}\KeywordTok{matrix}\NormalTok{(}\KeywordTok{vector}\NormalTok{(), }\DecValTok{0}\NormalTok{, }\DecValTok{19}\NormalTok{,}
                \DataTypeTok{dimnames=}\KeywordTok{list}\NormalTok{(}\KeywordTok{c}\NormalTok{(), }\KeywordTok{c}\NormalTok{(}\StringTok{"Date"}\NormalTok{, }\StringTok{"Is Prior"}\NormalTok{, }\StringTok{"Name"}\NormalTok{, }\StringTok{"Street 1"}\NormalTok{, }\StringTok{"Street 2"}\NormalTok{, }\StringTok{"City"}\NormalTok{, }\StringTok{"State"}\NormalTok{, }\StringTok{"Full Zip"}\NormalTok{, }
                                     \StringTok{"Country Name"}\NormalTok{, }\StringTok{"Outside US Postal Code"}\NormalTok{, }\StringTok{"Profession"}\NormalTok{, }\StringTok{"Employers Name"}\NormalTok{, }\StringTok{"Purpose"}\NormalTok{, }
                                     \StringTok{"Receipt Type Desc"}\NormalTok{, }\StringTok{"Account Abbr"}\NormalTok{, }\StringTok{"Form Of Payment Desc"}\NormalTok{, }\StringTok{"Description"}\NormalTok{, }\StringTok{"Amount"}\NormalTok{, }
                                     \StringTok{"Sum To Date"}\NormalTok{))),}
                \DataTypeTok{stringsAsFactors=}\NormalTok{F)}
  
    \CommentTok{# Constructs CSV paths from district and last name}
\NormalTok{    urls <-}\StringTok{ }\KeywordTok{c}\NormalTok{(}\KeywordTok{sprintf}\NormalTok{(}\StringTok{"../python/data/senate2018/%s/%s/q1.csv"}\NormalTok{, district, lastname),}
              \KeywordTok{sprintf}\NormalTok{(}\StringTok{"../python/data/senate2018/%s/%s/q2.csv"}\NormalTok{, district, lastname),}
              \KeywordTok{sprintf}\NormalTok{(}\StringTok{"../python/data/senate2018/%s/%s/q3.csv"}\NormalTok{, district, lastname),}
              \KeywordTok{sprintf}\NormalTok{(}\StringTok{"../python/data/senate2018/%s/%s/q4.csv"}\NormalTok{, district, lastname))}
    
    \CommentTok{# Tries to open each quarterly receipt CSV and unions it to a running dataframe}
    \ControlFlowTok{for}\NormalTok{(url }\ControlFlowTok{in}\NormalTok{ urls)\{}
      \CommentTok{# If the file doesn't exist, skip it}
      \ControlFlowTok{if}\NormalTok{ (}\KeywordTok{file.exists}\NormalTok{(url)) \{}
\NormalTok{        temp <-}\StringTok{ }\KeywordTok{read_csv}\NormalTok{(url, }
          \DataTypeTok{col_types =} \KeywordTok{cols}\NormalTok{(}
            \StringTok{`}\DataTypeTok{Account Abbr}\StringTok{`}\NormalTok{ =}\StringTok{ }\KeywordTok{col_character}\NormalTok{(), }
              \DataTypeTok{City =} \KeywordTok{col_character}\NormalTok{(), }
            \StringTok{`}\DataTypeTok{Country Name}\StringTok{`}\NormalTok{ =}\StringTok{ }\KeywordTok{col_character}\NormalTok{(), }
              \DataTypeTok{Date =} \KeywordTok{col_date}\NormalTok{(}\DataTypeTok{format =} \StringTok{"%m/%d/%Y"}\NormalTok{), }
              \DataTypeTok{Description =} \KeywordTok{col_character}\NormalTok{(), }
            \StringTok{`}\DataTypeTok{Employers Name}\StringTok{`}\NormalTok{ =}\StringTok{ }\KeywordTok{col_character}\NormalTok{(), }
              \StringTok{`}\DataTypeTok{Full Zip}\StringTok{`}\NormalTok{ =}\StringTok{ }\KeywordTok{col_character}\NormalTok{(), }
            \StringTok{`}\DataTypeTok{Outside US Postal Code}\StringTok{`}\NormalTok{ =}\StringTok{ }\KeywordTok{col_character}\NormalTok{(), }
              \DataTypeTok{Profession =} \KeywordTok{col_character}\NormalTok{(), }
            \DataTypeTok{Purpose =} \KeywordTok{col_character}\NormalTok{(), }
              \DataTypeTok{State =} \KeywordTok{col_character}\NormalTok{(), }
            \StringTok{`}\DataTypeTok{Street 1}\StringTok{`}\NormalTok{ =}\StringTok{ }\KeywordTok{col_character}\NormalTok{(), }
              \StringTok{`}\DataTypeTok{Street 2}\StringTok{`}\NormalTok{ =}\StringTok{ }\KeywordTok{col_character}\NormalTok{()),}
          \DataTypeTok{skip =} \DecValTok{1}\NormalTok{)}
        
\NormalTok{        rcpts_complete <-}\StringTok{ }\KeywordTok{union_all}\NormalTok{(rcpts_complete, temp)}
\NormalTok{      \}}
\NormalTok{    \}}
    
    \CommentTok{#Finally, we should rename the columns to remove spaces and generally promote brevity.}
    \KeywordTok{names}\NormalTok{(rcpts_complete) <-}\StringTok{ }\KeywordTok{c}\NormalTok{(}\StringTok{"date"}\NormalTok{,}\StringTok{"prior"}\NormalTok{,}\StringTok{"donor"}\NormalTok{,}\StringTok{"street1"}\NormalTok{,}\StringTok{"street2"}\NormalTok{,}\StringTok{"city"}\NormalTok{,}\StringTok{"state"}\NormalTok{,}\StringTok{"zip"}\NormalTok{,}\StringTok{"country"}\NormalTok{,}\StringTok{"postal"}\NormalTok{,}\StringTok{"profession"}\NormalTok{,}\StringTok{"employer"}\NormalTok{,}\StringTok{"purpose"}\NormalTok{,}\StringTok{"type"}\NormalTok{,}\StringTok{"account"}\NormalTok{,}\StringTok{"payment_form"}\NormalTok{,}\StringTok{"description"}\NormalTok{,}\StringTok{"amount"}\NormalTok{,}\StringTok{"sum_to_date"}\NormalTok{)}
    
    \KeywordTok{return}\NormalTok{(rcpts_complete)}
\NormalTok{\}}

\CommentTok{#candidateData <- candidateData %>%}
\CommentTok{#  mutate(receipts2018 = list(consolidateReceipts(district, tolower(lastname))))}

\CommentTok{# For each candidate, consolidate their quarterly receipts}
\ControlFlowTok{for}\NormalTok{ (row }\ControlFlowTok{in} \DecValTok{1}\OperatorTok{:}\KeywordTok{nrow}\NormalTok{(candidateData)) \{}
\NormalTok{    district <-}\StringTok{ }\NormalTok{candidateData[row, }\StringTok{"district"}\NormalTok{]}
\NormalTok{    lastname <-}\StringTok{ }\NormalTok{candidateData[row, }\StringTok{"lastname"}\NormalTok{] }\OperatorTok\StringTok{ }\KeywordTok{tolower}\NormalTok{()}
    
\NormalTok{    rcpts_complete <-}\StringTok{ }\KeywordTok{consolidateReceipts}\NormalTok{(district, lastname)}

\NormalTok{    receiptsAll[[row]] <-}\StringTok{ }\NormalTok{rcpts_complete}
\NormalTok{\}}

\CommentTok{# Update the candidate data with their receipts as lists}
\NormalTok{candidateData}\OperatorTok{$}\NormalTok{receipts2018 <-}\StringTok{ }\NormalTok{receiptsAll}
\end{Highlighting}
\end{Shaded}

\hypertarget{calculate-self-funding}{%
\subsubsection{Calculate self funding}\label{calculate-self-funding}}

\hypertarget{searches-through-each-receipt-data-frame-for-matches-with-the-candidates-name}{%
\paragraph{Searches through each receipt data frame for matches with the
candidate's
name}\label{searches-through-each-receipt-data-frame-for-matches-with-the-candidates-name}}

\begin{Shaded}
\begin{Highlighting}[]
\CommentTok{# Given a dataframe of receipts, a last name, and a first name/aliases, calculates the amount of self funding a candidate received}
\CommentTok{# Returns the total amount received through self funding}
\NormalTok{calculateSelfFunding <-}\StringTok{ }\ControlFlowTok{function}\NormalTok{(receipts, lastname, firstnames) \{}
  \CommentTok{# Regex pattern that matches any instance where the candidate's first name/aliases precedes their last name, with anything in between and at the beginning and end of strings}
\NormalTok{  pattern <-}\StringTok{ }\KeywordTok{sprintf}\NormalTok{(}\StringTok{".*%s.*%s.*"}\NormalTok{, }\KeywordTok{paste}\NormalTok{(firstnames, }\DataTypeTok{collapse =} \StringTok{"|"}\NormalTok{), lastname)}
  
  \CommentTok{# Filters by using grepping with the above pattern on the donors}
\NormalTok{  receipts <-}\StringTok{ }\NormalTok{receipts }\OperatorTok
\StringTok{    }\KeywordTok{filter}\NormalTok{(}\KeywordTok{grepl}\NormalTok{(pattern, }\KeywordTok{c}\NormalTok{(receipts2018.donor), }\DataTypeTok{ignore.case=}\OtherTok{TRUE}\NormalTok{))}
  
  \CommentTok{# Uncomment this print statement to check for discrepancy in resulting list of self-funded receipts}
  \CommentTok{# print(receipts)}
  
\NormalTok{  total <-}\StringTok{ }\KeywordTok{sum}\NormalTok{(receipts}\OperatorTok{$}\NormalTok{receipts2018.amount)}
  \KeywordTok{return}\NormalTok{(total)}
\NormalTok{\}}

\NormalTok{selfFundedAll <-}\StringTok{ }\KeywordTok{c}\NormalTok{()}

\CommentTok{# For each candidate, calculate their self-funded total}
\ControlFlowTok{for}\NormalTok{ (row }\ControlFlowTok{in} \DecValTok{1}\OperatorTok{:}\KeywordTok{nrow}\NormalTok{(candidateData)) \{}
\NormalTok{    district <-}\StringTok{ }\NormalTok{candidateData[row, }\StringTok{"district"}\NormalTok{]}
\NormalTok{    lastname <-}\StringTok{ }\NormalTok{candidateData[row, }\StringTok{"lastname"}\NormalTok{] }\OperatorTok\StringTok{ }\KeywordTok{tolower}\NormalTok{()}
\NormalTok{    firstnames <-}\StringTok{ }\KeywordTok{c}\NormalTok{(candidateData[row, }\StringTok{"firstname"}\NormalTok{] }\OperatorTok\StringTok{ }\KeywordTok{tolower}\NormalTok{())}
\NormalTok{    aliases <-}\StringTok{ }\NormalTok{candidateData[row, }\StringTok{"aliases"}\NormalTok{]}
    \ControlFlowTok{if}\NormalTok{ (}\OperatorTok{!}\NormalTok{(}\KeywordTok{is.na}\NormalTok{(aliases) }\OperatorTok{||}\StringTok{ }\NormalTok{aliases }\OperatorTok{==}\StringTok{ ""}\NormalTok{) }\OperatorTok{&&}\StringTok{ }\KeywordTok{typeof}\NormalTok{(aliases) }\OperatorTok{==}\StringTok{ "character"}\NormalTok{) \{}
\NormalTok{      firstnames <-}\StringTok{ }\KeywordTok{append}\NormalTok{(firstnames, }\KeywordTok{strsplit}\NormalTok{(candidateData[row, }\StringTok{"aliases"}\NormalTok{], }\StringTok{"/"}\NormalTok{))}
\NormalTok{    \}}
\NormalTok{    receipts <-}\StringTok{ }\NormalTok{candidateData[row, }\StringTok{"receipts2018"}\NormalTok{]}
\NormalTok{    receipts <-}\StringTok{ }\KeywordTok{do.call}\NormalTok{(data.frame, receipts)}
    
\NormalTok{    selfFundedAmt <-}\StringTok{ }\KeywordTok{calculateSelfFunding}\NormalTok{(receipts, lastname, firstnames)}

\NormalTok{    selfFundedAll <-}\StringTok{ }\KeywordTok{append}\NormalTok{(selfFundedAll, selfFundedAmt)}
\NormalTok{\}}

\CommentTok{# Update the candidate data with their self-funded totals}
\NormalTok{candidateData}\OperatorTok{$}\NormalTok{selfFundedAmt <-}\StringTok{ }\NormalTok{selfFundedAll}
\end{Highlighting}
\end{Shaded}

\hypertarget{calculate-relative-self-funding-compared-to-opponent}{%
\subsubsection{Calculate relative self funding compared to
opponent}\label{calculate-relative-self-funding-compared-to-opponent}}

\hypertarget{for-candidate-well-get-the-relative-funding-compared-to-their-district-opponent-in-the-other-party}{%
\paragraph{For candidate, we'll get the relative funding compared to
their district opponent in the other
party}\label{for-candidate-well-get-the-relative-funding-compared-to-their-district-opponent-in-the-other-party}}

\begin{Shaded}
\begin{Highlighting}[]
\NormalTok{candidateDataNaN <-}\StringTok{ }\NormalTok{candidateData }\OperatorTok
\StringTok{  }\KeywordTok{group_by}\NormalTok{(district) }\OperatorTok
\StringTok{  }\KeywordTok{mutate}\NormalTok{(}\DataTypeTok{selfFundedTotal =} \KeywordTok{sum}\NormalTok{(selfFundedAmt), }\DataTypeTok{selfFundedRelative =}\NormalTok{ selfFundedAmt }\OperatorTok{/}\StringTok{ }\NormalTok{(selfFundedTotal)) }

\NormalTok{candidateDataClean <-}\StringTok{ }\NormalTok{candidateDataNaN }\OperatorTok
\StringTok{  }\KeywordTok{mutate}\NormalTok{(}\DataTypeTok{selfFundedRelative =} \ControlFlowTok{if}\NormalTok{(}\OperatorTok{!}\KeywordTok{is.finite}\NormalTok{(selfFundedRelative)[}\DecValTok{1}\NormalTok{]) }\FloatTok{0.5} \ControlFlowTok{else}\NormalTok{ selfFundedRelative)}
\end{Highlighting}
\end{Shaded}

\hypertarget{calculate-the-percentage-of-campaigns-won-by-the-highest-self-funder}{%
\subsubsection{Calculate the percentage of campaigns won by the highest
self
funder}\label{calculate-the-percentage-of-campaigns-won-by-the-highest-self-funder}}

\begin{Shaded}
\begin{Highlighting}[]
\KeywordTok{print}\NormalTok{(}\StringTok{"Winner's who self funded more than their opponent:"}\NormalTok{)}
\end{Highlighting}
\end{Shaded}

\begin{verbatim}
## [1] "Winner's who self funded more than their opponent:"
\end{verbatim}

\begin{Shaded}
\begin{Highlighting}[]
\KeywordTok{nrow}\NormalTok{(candidateDataClean }\OperatorTok
\StringTok{  }\KeywordTok{filter}\NormalTok{(win }\OperatorTok{==}\StringTok{ }\DecValTok{1}\NormalTok{, selfFundedRelative }\OperatorTok{>}\StringTok{ }\FloatTok{0.5}\NormalTok{))}
\end{Highlighting}
\end{Shaded}

\begin{verbatim}
## [1] 15
\end{verbatim}

\begin{Shaded}
\begin{Highlighting}[]
\KeywordTok{print}\NormalTok{(}\StringTok{"Winner's who self funded less than their opponent:"}\NormalTok{)}
\end{Highlighting}
\end{Shaded}

\begin{verbatim}
## [1] "Winner's who self funded less than their opponent:"
\end{verbatim}

\begin{Shaded}
\begin{Highlighting}[]
\KeywordTok{nrow}\NormalTok{(candidateDataClean }\OperatorTok
\StringTok{  }\KeywordTok{filter}\NormalTok{(win }\OperatorTok{==}\StringTok{ }\DecValTok{1}\NormalTok{, selfFundedRelative }\OperatorTok{<}\StringTok{ }\FloatTok{0.5}\NormalTok{))}
\end{Highlighting}
\end{Shaded}

\begin{verbatim}
## [1] 19
\end{verbatim}

\hypertarget{plot-relative-self-funding-vs-margin-of-victory-for-winners}{%
\subsubsection{Plot relative self funding vs margin of victory for
winners}\label{plot-relative-self-funding-vs-margin-of-victory-for-winners}}

\begin{Shaded}
\begin{Highlighting}[]
\NormalTok{candidateDataClean }\OperatorTok
\StringTok{  }\KeywordTok{filter}\NormalTok{(win }\OperatorTok{==}\StringTok{ }\DecValTok{1}\NormalTok{) }\OperatorTok
\StringTok{  }\KeywordTok{ggplot}\NormalTok{(}\DataTypeTok{mapping =} \KeywordTok{aes}\NormalTok{(}\DataTypeTok{x =}\NormalTok{ selfFundedRelative, }\DataTypeTok{y =}\NormalTok{ margin, }\DataTypeTok{size =}\NormalTok{ selfFundedAmt)) }\OperatorTok{+}\StringTok{ }
\StringTok{  }\KeywordTok{geom_point}\NormalTok{() }\OperatorTok{+}
\StringTok{  }\KeywordTok{ggtitle}\NormalTok{(}\StringTok{"Self Funding vs Margin of Victory"}\NormalTok{) }\OperatorTok{+}
\StringTok{  }\KeywordTok{xlab}\NormalTok{(}\StringTok{"Relative Self Funding"}\NormalTok{) }\OperatorTok{+}\StringTok{ }
\StringTok{  }\KeywordTok{ylab}\NormalTok{(}\StringTok{"Margin of Victory (percentage)"}\NormalTok{) }\OperatorTok{+}
\StringTok{  }\KeywordTok{labs}\NormalTok{(}
    \DataTypeTok{size =} \StringTok{"Self Funding ($)"}
\NormalTok{  ) }\OperatorTok{+}
\StringTok{  }\KeywordTok{geom_smooth}\NormalTok{()}
\end{Highlighting}
\end{Shaded}

\begin{verbatim}
## `geom_smooth()` using method = 'loess' and formula 'y ~ x'
\end{verbatim}

\includegraphics{final-story_files/figure-latex/unnamed-chunk-6-1.pdf}

\begin{Shaded}
\begin{Highlighting}[]
\NormalTok{candidateDataClean }\OperatorTok
\StringTok{  }\KeywordTok{filter}\NormalTok{(win }\OperatorTok{==}\StringTok{ }\DecValTok{1}\NormalTok{, selfFundedTotal }\OperatorTok{>}\StringTok{ }\DecValTok{1000}\NormalTok{) }\OperatorTok
\StringTok{  }\KeywordTok{ggplot}\NormalTok{(}\DataTypeTok{mapping =} \KeywordTok{aes}\NormalTok{(}\DataTypeTok{x =}\NormalTok{ selfFundedRelative, }\DataTypeTok{y =}\NormalTok{ margin, }\DataTypeTok{size =}\NormalTok{ selfFundedAmt)) }\OperatorTok{+}\StringTok{ }
\StringTok{  }\KeywordTok{geom_point}\NormalTok{() }\OperatorTok{+}\StringTok{ }
\StringTok{  }\KeywordTok{ggtitle}\NormalTok{(}\StringTok{"Self Funding vs Margin of Victory"}\NormalTok{) }\OperatorTok{+}
\StringTok{  }\KeywordTok{xlab}\NormalTok{(}\StringTok{"Relative Self Funding"}\NormalTok{) }\OperatorTok{+}\StringTok{ }
\StringTok{  }\KeywordTok{ylab}\NormalTok{(}\StringTok{"Margin of Victory (percentage)"}\NormalTok{) }\OperatorTok{+}
\StringTok{  }\KeywordTok{labs}\NormalTok{(}
    \DataTypeTok{size =} \StringTok{"Self Funding ($)"}
\NormalTok{  ) }\OperatorTok{+}
\StringTok{  }\KeywordTok{geom_smooth}\NormalTok{()}
\end{Highlighting}
\end{Shaded}

\begin{verbatim}
## `geom_smooth()` using method = 'loess' and formula 'y ~ x'
\end{verbatim}

\includegraphics{final-story_files/figure-latex/unnamed-chunk-6-2.pdf}

\hypertarget{remove-all-races-where-neither-party-self-funded}{%
\subsubsection{Remove all races where neither party self
funded}\label{remove-all-races-where-neither-party-self-funded}}

\begin{Shaded}
\begin{Highlighting}[]
\NormalTok{candidateDataNaN }\OperatorTok
\StringTok{  }\KeywordTok{filter}\NormalTok{(win }\OperatorTok{==}\StringTok{ }\DecValTok{1}\NormalTok{, }\KeywordTok{is.finite}\NormalTok{(selfFundedRelative)[}\DecValTok{1}\NormalTok{]) }\OperatorTok
\StringTok{  }\KeywordTok{ggplot}\NormalTok{(}\DataTypeTok{mapping =} \KeywordTok{aes}\NormalTok{(}\DataTypeTok{x =}\NormalTok{ selfFundedRelative, }\DataTypeTok{y =}\NormalTok{ margin, }\DataTypeTok{size =}\NormalTok{ selfFundedAmt)) }\OperatorTok{+}\StringTok{ }
\StringTok{  }\KeywordTok{geom_point}\NormalTok{() }\OperatorTok{+}\StringTok{ }
\StringTok{  }\KeywordTok{ggtitle}\NormalTok{(}\StringTok{"Self Funding vs Margin of Victory"}\NormalTok{) }\OperatorTok{+}
\StringTok{  }\KeywordTok{xlab}\NormalTok{(}\StringTok{"Relative Self Funding"}\NormalTok{) }\OperatorTok{+}\StringTok{ }
\StringTok{  }\KeywordTok{ylab}\NormalTok{(}\StringTok{"Margin of Victory (percentage)"}\NormalTok{) }\OperatorTok{+}
\StringTok{  }\KeywordTok{labs}\NormalTok{(}
    \DataTypeTok{size =} \StringTok{"Self Funding ($)"}
\NormalTok{  ) }\OperatorTok{+}
\StringTok{  }\KeywordTok{geom_smooth}\NormalTok{()}
\end{Highlighting}
\end{Shaded}

\begin{verbatim}
## `geom_smooth()` using method = 'loess' and formula 'y ~ x'
\end{verbatim}

\includegraphics{final-story_files/figure-latex/unnamed-chunk-7-1.pdf}

\begin{Shaded}
\begin{Highlighting}[]
\KeywordTok{cor}\NormalTok{(candidateDataClean}\OperatorTok{$}\NormalTok{selfFundedRelative, candidateDataClean}\OperatorTok{$}\NormalTok{margin)}
\end{Highlighting}
\end{Shaded}

\begin{verbatim}
## [1] 5.649065e-18
\end{verbatim}

\hypertarget{gather-close-funding-data}{%
\subsubsection{Gather close funding
data}\label{gather-close-funding-data}}

\hypertarget{get-data-donated-from-family-and-family-businesses}{%
\paragraph{Get data donated from family and family
businesses}\label{get-data-donated-from-family-and-family-businesses}}

\begin{Shaded}
\begin{Highlighting}[]
\NormalTok{candidateDataClose <-}\StringTok{ }\NormalTok{candidateData}

\CommentTok{# Given a dataframe of receipts, a last name, and a first name/aliases, calculates the amount of self funding a candidate received}
\CommentTok{# Returns the total amount received through self funding}
\NormalTok{calculateCloseFunding <-}\StringTok{ }\ControlFlowTok{function}\NormalTok{(receipts, lastname) \{}
  \CommentTok{# Regex pattern that matches any instance where the candidate's first name/aliases precedes their last name, with anything in between and at the beginning and end of strings}
\NormalTok{  pattern <-}\StringTok{ }\KeywordTok{sprintf}\NormalTok{(}\StringTok{".*%s.*"}\NormalTok{, lastname)}
  
  \CommentTok{# Filters by using grepping with the above pattern on the donors}
\NormalTok{  receipts <-}\StringTok{ }\NormalTok{receipts[ }\KeywordTok{grepl}\NormalTok{(pattern, receipts}\OperatorTok{$}\NormalTok{receipts2018.donor, }\DataTypeTok{ignore.case=}\OtherTok{TRUE}\NormalTok{) }\OperatorTok{|}\StringTok{ }\KeywordTok{grepl}\NormalTok{(pattern, receipts}\OperatorTok{$}\NormalTok{receipts2018.employer, }\DataTypeTok{ignore.case=}\OtherTok{TRUE}\NormalTok{), ]}
  
  \CommentTok{# Uncomment this print statement to check for discrepancy in resulting list of self-funded receipts}
  \CommentTok{# print(receipts)}
  
\NormalTok{  total <-}\StringTok{ }\KeywordTok{sum}\NormalTok{(receipts}\OperatorTok{$}\NormalTok{receipts2018.amount)}
  \KeywordTok{return}\NormalTok{(total)}
\NormalTok{\}}

\NormalTok{closeFundedAll <-}\StringTok{ }\KeywordTok{c}\NormalTok{()}

\CommentTok{# For each candidate, calculate their self-funded total}
\ControlFlowTok{for}\NormalTok{ (row }\ControlFlowTok{in} \DecValTok{1}\OperatorTok{:}\KeywordTok{nrow}\NormalTok{(candidateData)) \{}
\NormalTok{    district <-}\StringTok{ }\NormalTok{candidateData[row, }\StringTok{"district"}\NormalTok{]}
\NormalTok{    lastname <-}\StringTok{ }\NormalTok{candidateData[row, }\StringTok{"lastname"}\NormalTok{] }\OperatorTok\StringTok{ }\KeywordTok{tolower}\NormalTok{()}
\NormalTok{    receipts <-}\StringTok{ }\NormalTok{candidateData[row, }\StringTok{"receipts2018"}\NormalTok{]}
\NormalTok{    receipts <-}\StringTok{ }\KeywordTok{do.call}\NormalTok{(data.frame, receipts)}
    
\NormalTok{    closeFundedAmt <-}\StringTok{ }\KeywordTok{calculateCloseFunding}\NormalTok{(receipts, lastname)}

\NormalTok{    closeFundedAll <-}\StringTok{ }\KeywordTok{append}\NormalTok{(closeFundedAll, closeFundedAmt)}
\NormalTok{\}}

\CommentTok{# Update the candidate data with their self-funded totals}
\NormalTok{candidateDataClose}\OperatorTok{$}\NormalTok{closeFundedAmt <-}\StringTok{ }\NormalTok{closeFundedAll}

\NormalTok{candidateDataClose}\OperatorTok{$}\NormalTok{closeFundedAmt}
\end{Highlighting}
\end{Shaded}

\begin{verbatim}
##  [1]   6865.29   2935.33    751.00      0.00    100.00   1200.00      0.00
##  [8]      0.00    215.00      0.00   1707.74   3000.00    700.00    100.00
## [15]   1500.00      0.00   5350.00    950.00     50.00   1000.00   1600.00
## [22]      0.00   6386.48      0.00   1182.61   2000.00  25281.77    100.00
## [29]   2000.00      0.00  50432.11      0.00 105000.00    207.00  58319.49
## [36]      0.00   1369.00    500.00  26826.71   1500.00    207.00   2095.00
## [43]      0.00    100.00     62.00      0.00  58828.97    400.00   4460.62
## [50]   3603.77   1457.00      0.00  38397.00      0.00   3194.55    747.70
## [57]    150.00      0.00      0.00      0.00    861.95      0.00    100.00
## [64]      0.00   2335.00   1493.66   8706.41   1500.00    750.00   1326.00
## [71]   6700.00      0.00      0.00      0.00    809.74  15524.28    715.10
## [78]   5200.00   6973.00    150.00    950.00   5000.00      0.00   1702.75
## [85]      0.00      0.00   2350.00   3000.00    125.00   2728.05      0.00
## [92]   2576.89  55936.90    500.00  17021.19    449.07
\end{verbatim}

\hypertarget{calculate-relative-close-funding-compared-to-opponent}{%
\subsubsection{Calculate relative close funding compared to
opponent}\label{calculate-relative-close-funding-compared-to-opponent}}

\begin{Shaded}
\begin{Highlighting}[]
\NormalTok{candidateDataCloseNaN <-}\StringTok{ }\NormalTok{candidateDataClose }\OperatorTok
\StringTok{  }\KeywordTok{group_by}\NormalTok{(district) }\OperatorTok
\StringTok{  }\KeywordTok{mutate}\NormalTok{(}\DataTypeTok{closeFundedTotal =} \KeywordTok{sum}\NormalTok{(closeFundedAmt), }\DataTypeTok{closeFundedRelative =}\NormalTok{ closeFundedAmt }\OperatorTok{/}\StringTok{ }\NormalTok{(closeFundedTotal)) }

\NormalTok{candidateDataCloseClean <-}\StringTok{ }\NormalTok{candidateDataCloseNaN }\OperatorTok
\StringTok{  }\KeywordTok{mutate}\NormalTok{(}\DataTypeTok{closeFundedRelative =} \ControlFlowTok{if}\NormalTok{(}\OperatorTok{!}\NormalTok{(}\KeywordTok{is.finite}\NormalTok{(closeFundedRelative))[}\DecValTok{1}\NormalTok{]) }\FloatTok{0.5} \ControlFlowTok{else}\NormalTok{ closeFundedRelative)}
\end{Highlighting}
\end{Shaded}

\hypertarget{calculate-the-percentage-of-campaigns-won-by-the-highest-close-funder}{%
\subsubsection{Calculate the percentage of campaigns won by the highest
close
funder}\label{calculate-the-percentage-of-campaigns-won-by-the-highest-close-funder}}

\begin{Shaded}
\begin{Highlighting}[]
\KeywordTok{print}\NormalTok{(}\StringTok{"Winner's who close funded more than their opponent:"}\NormalTok{)}
\end{Highlighting}
\end{Shaded}

\begin{verbatim}
## [1] "Winner's who close funded more than their opponent:"
\end{verbatim}

\begin{Shaded}
\begin{Highlighting}[]
\KeywordTok{nrow}\NormalTok{(candidateDataCloseClean }\OperatorTok
\StringTok{  }\KeywordTok{filter}\NormalTok{(win }\OperatorTok{==}\StringTok{ }\DecValTok{1}\NormalTok{, closeFundedRelative }\OperatorTok{>}\StringTok{ }\FloatTok{0.5}\NormalTok{))}
\end{Highlighting}
\end{Shaded}

\begin{verbatim}
## [1] 22
\end{verbatim}

\begin{Shaded}
\begin{Highlighting}[]
\KeywordTok{print}\NormalTok{(}\StringTok{"Winner's who close funded less than their opponent:"}\NormalTok{)}
\end{Highlighting}
\end{Shaded}

\begin{verbatim}
## [1] "Winner's who close funded less than their opponent:"
\end{verbatim}

\begin{Shaded}
\begin{Highlighting}[]
\KeywordTok{nrow}\NormalTok{(candidateDataCloseClean }\OperatorTok
\StringTok{  }\KeywordTok{filter}\NormalTok{(win }\OperatorTok{==}\StringTok{ }\DecValTok{1}\NormalTok{, closeFundedRelative }\OperatorTok{<}\StringTok{ }\FloatTok{0.5}\NormalTok{))}
\end{Highlighting}
\end{Shaded}

\begin{verbatim}
## [1] 22
\end{verbatim}

\hypertarget{plot-relative-funding-vs-margin-of-victory-for-winners}{%
\subsubsection{Plot relative funding vs margin of victory for
winners}\label{plot-relative-funding-vs-margin-of-victory-for-winners}}

\begin{Shaded}
\begin{Highlighting}[]
\NormalTok{candidateDataCloseClean }\OperatorTok
\StringTok{  }\KeywordTok{filter}\NormalTok{(win }\OperatorTok{==}\StringTok{ }\DecValTok{1}\NormalTok{) }\OperatorTok
\StringTok{  }\KeywordTok{ggplot}\NormalTok{(}\DataTypeTok{mapping =} \KeywordTok{aes}\NormalTok{(}\DataTypeTok{x =}\NormalTok{ closeFundedRelative, }\DataTypeTok{y =}\NormalTok{ margin, }\DataTypeTok{size =}\NormalTok{ closeFundedAmt)) }\OperatorTok{+}\StringTok{ }
\StringTok{  }\KeywordTok{geom_point}\NormalTok{() }\OperatorTok{+}\StringTok{ }
\StringTok{  }\KeywordTok{ggtitle}\NormalTok{(}\StringTok{"Close Funding vs Margin of Victory"}\NormalTok{) }\OperatorTok{+}
\StringTok{  }\KeywordTok{xlab}\NormalTok{(}\StringTok{"Relative Close Funding"}\NormalTok{) }\OperatorTok{+}\StringTok{ }
\StringTok{  }\KeywordTok{ylab}\NormalTok{(}\StringTok{"Margin of Victory (percentage)"}\NormalTok{) }\OperatorTok{+}
\StringTok{  }\KeywordTok{labs}\NormalTok{(}
    \DataTypeTok{size =} \StringTok{"Close Funding ($)"}
\NormalTok{  ) }\OperatorTok{+}
\StringTok{  }\KeywordTok{geom_smooth}\NormalTok{()}
\end{Highlighting}
\end{Shaded}

\begin{verbatim}
## `geom_smooth()` using method = 'loess' and formula 'y ~ x'
\end{verbatim}

\includegraphics{final-story_files/figure-latex/unnamed-chunk-11-1.pdf}

\begin{Shaded}
\begin{Highlighting}[]
\NormalTok{candidateDataCloseClean }\OperatorTok
\StringTok{  }\KeywordTok{filter}\NormalTok{(win }\OperatorTok{==}\StringTok{ }\DecValTok{1}\NormalTok{, closeFundedTotal }\OperatorTok{>}\StringTok{ }\DecValTok{1000}\NormalTok{) }\OperatorTok
\StringTok{  }\KeywordTok{ggplot}\NormalTok{(}\DataTypeTok{mapping =} \KeywordTok{aes}\NormalTok{(}\DataTypeTok{x =}\NormalTok{ closeFundedRelative, }\DataTypeTok{y =}\NormalTok{ margin, }\DataTypeTok{size =}\NormalTok{ closeFundedAmt)) }\OperatorTok{+}\StringTok{ }
\StringTok{  }\KeywordTok{geom_point}\NormalTok{() }\OperatorTok{+}\StringTok{ }
\StringTok{  }\KeywordTok{ggtitle}\NormalTok{(}\StringTok{"Close Funding vs Margin of Victory"}\NormalTok{) }\OperatorTok{+}
\StringTok{  }\KeywordTok{xlab}\NormalTok{(}\StringTok{"Relative Close Funding"}\NormalTok{) }\OperatorTok{+}\StringTok{ }
\StringTok{  }\KeywordTok{ylab}\NormalTok{(}\StringTok{"Margin of Victory (percentage)"}\NormalTok{) }\OperatorTok{+}
\StringTok{  }\KeywordTok{labs}\NormalTok{(}
    \DataTypeTok{size =} \StringTok{"Close Funding ($)"}
\NormalTok{  ) }\OperatorTok{+}
\StringTok{  }\KeywordTok{geom_smooth}\NormalTok{()}
\end{Highlighting}
\end{Shaded}

\begin{verbatim}
## `geom_smooth()` using method = 'loess' and formula 'y ~ x'
\end{verbatim}

\includegraphics{final-story_files/figure-latex/unnamed-chunk-11-2.pdf}

\hypertarget{remove-all-races-where-neither-party-is-close-funded}{%
\subsubsection{Remove all races where neither party is close
funded}\label{remove-all-races-where-neither-party-is-close-funded}}

\begin{Shaded}
\begin{Highlighting}[]
\NormalTok{candidateDataCloseNaN }\OperatorTok
\StringTok{  }\KeywordTok{filter}\NormalTok{(win }\OperatorTok{==}\StringTok{ }\DecValTok{1}\NormalTok{, }\KeywordTok{is.finite}\NormalTok{(closeFundedRelative)[}\DecValTok{1}\NormalTok{]) }\OperatorTok
\StringTok{  }\KeywordTok{ggplot}\NormalTok{(}\DataTypeTok{mapping =} \KeywordTok{aes}\NormalTok{(}\DataTypeTok{x =}\NormalTok{ closeFundedRelative, }\DataTypeTok{y =}\NormalTok{ margin, }\DataTypeTok{size =}\NormalTok{ closeFundedAmt)) }\OperatorTok{+}\StringTok{ }
\StringTok{  }\KeywordTok{geom_point}\NormalTok{() }\OperatorTok{+}\StringTok{ }
\StringTok{  }\KeywordTok{ggtitle}\NormalTok{(}\StringTok{"Close Funding vs Margin of Victory"}\NormalTok{) }\OperatorTok{+}
\StringTok{  }\KeywordTok{xlab}\NormalTok{(}\StringTok{"Relative Close Funding"}\NormalTok{) }\OperatorTok{+}\StringTok{ }
\StringTok{  }\KeywordTok{ylab}\NormalTok{(}\StringTok{"Margin of Victory (percentage)"}\NormalTok{) }\OperatorTok{+}
\StringTok{  }\KeywordTok{labs}\NormalTok{(}
    \DataTypeTok{size =} \StringTok{"Close Funding ($)"}
\NormalTok{  ) }\OperatorTok{+}
\StringTok{  }\KeywordTok{geom_smooth}\NormalTok{()}
\end{Highlighting}
\end{Shaded}

\begin{verbatim}
## `geom_smooth()` using method = 'loess' and formula 'y ~ x'
\end{verbatim}

\includegraphics{final-story_files/figure-latex/unnamed-chunk-12-1.pdf}

\begin{Shaded}
\begin{Highlighting}[]
\KeywordTok{cor}\NormalTok{(candidateDataCloseClean}\OperatorTok{$}\NormalTok{closeFundedRelative, candidateDataCloseClean}\OperatorTok{$}\NormalTok{margin)}
\end{Highlighting}
\end{Shaded}

\begin{verbatim}
## [1] 8.218065e-20
\end{verbatim}

\hypertarget{calculate-self-funding-amounts-before-the-primary}{%
\subsubsection{Calculate self funding amounts before the
primary}\label{calculate-self-funding-amounts-before-the-primary}}

\begin{Shaded}
\begin{Highlighting}[]
\NormalTok{candidateDataEarly <-}\StringTok{ }\NormalTok{candidateData}

\CommentTok{# Given a dataframe of receipts, a last name, and a first name/aliases, calculates the amount of early self funding a candidate received}
\CommentTok{# Returns the total amount received through early self funding}
\NormalTok{calculateEarlySelfFunding <-}\StringTok{ }\ControlFlowTok{function}\NormalTok{(receipts, lastname, firstnames) \{}
  \CommentTok{# Regex pattern that matches any instance where the candidate's first name/aliases precedes their last name, with anything in between and at the beginning and end of strings}
\NormalTok{  pattern <-}\StringTok{ }\KeywordTok{sprintf}\NormalTok{(}\StringTok{".*%s.*%s.*"}\NormalTok{, }\KeywordTok{paste}\NormalTok{(firstnames, }\DataTypeTok{collapse =} \StringTok{"|"}\NormalTok{), lastname)}
  
  \CommentTok{# Filters by using grepping with the above pattern on the donors}
\NormalTok{  receipts <-}\StringTok{ }\NormalTok{receipts }\OperatorTok
\StringTok{    }\KeywordTok{filter}\NormalTok{(}\KeywordTok{grepl}\NormalTok{(pattern, }\KeywordTok{c}\NormalTok{(receipts2018.donor), }\DataTypeTok{ignore.case=}\OtherTok{TRUE}\NormalTok{), }\KeywordTok{ymd}\NormalTok{(receipts2018.date) }\OperatorTok{<}\StringTok{ }\KeywordTok{mdy}\NormalTok{(}\StringTok{"5/8/2018"}\NormalTok{))}
  
  \CommentTok{# Uncomment this print statement to check for discrepancy in resulting list of early self-funded receipts}
  \CommentTok{# print(receipts)}
  
\NormalTok{  total <-}\StringTok{ }\KeywordTok{sum}\NormalTok{(receipts}\OperatorTok{$}\NormalTok{receipts2018.amount)}
  \KeywordTok{return}\NormalTok{(total)}
\NormalTok{\}}

\NormalTok{earlySelfFundedAll <-}\StringTok{ }\KeywordTok{c}\NormalTok{()}

\CommentTok{# For each candidate, calculate their early self-funded total}
\ControlFlowTok{for}\NormalTok{ (row }\ControlFlowTok{in} \DecValTok{1}\OperatorTok{:}\KeywordTok{nrow}\NormalTok{(candidateData)) \{}
\NormalTok{    district <-}\StringTok{ }\NormalTok{candidateData[row, }\StringTok{"district"}\NormalTok{]}
\NormalTok{    lastname <-}\StringTok{ }\NormalTok{candidateData[row, }\StringTok{"lastname"}\NormalTok{] }\OperatorTok\StringTok{ }\KeywordTok{tolower}\NormalTok{()}
\NormalTok{    firstnames <-}\StringTok{ }\KeywordTok{c}\NormalTok{(candidateData[row, }\StringTok{"firstname"}\NormalTok{] }\OperatorTok\StringTok{ }\KeywordTok{tolower}\NormalTok{())}
    \CommentTok{# <- append(strsplit(candidateData[row, "aliases"], "/"), candidateData[row, "firstname"])}
\NormalTok{    receipts <-}\StringTok{ }\NormalTok{candidateData[row, }\StringTok{"receipts2018"}\NormalTok{]}
\NormalTok{    receipts <-}\StringTok{ }\KeywordTok{do.call}\NormalTok{(data.frame, receipts)}
    
\NormalTok{    earlySelfFundedAmt <-}\StringTok{ }\KeywordTok{calculateEarlySelfFunding}\NormalTok{(receipts, lastname, firstnames)}

\NormalTok{    earlySelfFundedAll <-}\StringTok{ }\KeywordTok{append}\NormalTok{(earlySelfFundedAll, earlySelfFundedAmt)}
\NormalTok{\}}

\CommentTok{# Update the candidate data with their self-funded totals}
\NormalTok{candidateDataEarly}\OperatorTok{$}\NormalTok{earlySelfFundedAmt <-}\StringTok{ }\NormalTok{earlySelfFundedAll}
\end{Highlighting}
\end{Shaded}

\hypertarget{calculate-relative-early-self-funding-compared-to-opponent}{%
\subsubsection{Calculate relative early self funding compared to
opponent}\label{calculate-relative-early-self-funding-compared-to-opponent}}

\hypertarget{for-candidate-well-get-the-relative-funding-compared-to-their-district-opponent-in-the-other-party-1}{%
\paragraph{For candidate, we'll get the relative funding compared to
their district opponent in the other
party}\label{for-candidate-well-get-the-relative-funding-compared-to-their-district-opponent-in-the-other-party-1}}

\begin{Shaded}
\begin{Highlighting}[]
\NormalTok{candidateDataEarlyNaN <-}\StringTok{ }\NormalTok{candidateDataEarly }\OperatorTok
\StringTok{  }\KeywordTok{group_by}\NormalTok{(district) }\OperatorTok
\StringTok{  }\KeywordTok{mutate}\NormalTok{(}\DataTypeTok{earlySelfFundedTotal =} \KeywordTok{sum}\NormalTok{(earlySelfFundedAmt), }\DataTypeTok{earlySelfFundedRelative =}\NormalTok{ earlySelfFundedAmt }\OperatorTok{/}\StringTok{ }\NormalTok{(earlySelfFundedTotal)) }

\NormalTok{candidateDataEarlyClean <-}\StringTok{ }\NormalTok{candidateDataEarlyNaN }\OperatorTok
\StringTok{  }\KeywordTok{mutate}\NormalTok{(}\DataTypeTok{earlySelfFundedRelative =} \ControlFlowTok{if}\NormalTok{(}\OperatorTok{!}\KeywordTok{is.finite}\NormalTok{(earlySelfFundedRelative)[}\DecValTok{1}\NormalTok{]) }\FloatTok{0.5} \ControlFlowTok{else}\NormalTok{ earlySelfFundedRelative)}

\ControlFlowTok{for}\NormalTok{ (row }\ControlFlowTok{in} \DecValTok{1}\OperatorTok{:}\KeywordTok{nrow}\NormalTok{(candidateDataEarlyNaN)) \{}
\NormalTok{  receipts <-}\StringTok{ }\NormalTok{candidateDataEarlyNaN[row, }\StringTok{"receipts2018"}\NormalTok{]}
\NormalTok{  receipts <-}\StringTok{ }\KeywordTok{do.call}\NormalTok{(data.frame, receipts)}
\NormalTok{  candidateDataEarlyNaN[row, }\StringTok{"totalAmount"}\NormalTok{] =}\StringTok{ }\KeywordTok{sum}\NormalTok{(receipts}\OperatorTok{$}\NormalTok{receipts2018.amount)}
\NormalTok{\}}

\ControlFlowTok{for}\NormalTok{ (row }\ControlFlowTok{in} \DecValTok{1}\OperatorTok{:}\KeywordTok{nrow}\NormalTok{(candidateDataEarlyClean)) \{}
\NormalTok{  receipts <-}\StringTok{ }\NormalTok{candidateDataEarlyClean[row, }\StringTok{"receipts2018"}\NormalTok{]}
\NormalTok{  receipts <-}\StringTok{ }\KeywordTok{do.call}\NormalTok{(data.frame, receipts)}
\NormalTok{  candidateDataEarlyClean[row, }\StringTok{"totalAmount"}\NormalTok{] =}\StringTok{ }\KeywordTok{sum}\NormalTok{(receipts}\OperatorTok{$}\NormalTok{receipts2018.amount)}
\NormalTok{\}}

\NormalTok{candidateDataEarlyNaN <-}\StringTok{ }\NormalTok{candidateDataEarlyNaN }\OperatorTok
\StringTok{  }\KeywordTok{group_by}\NormalTok{(district) }\OperatorTok
\StringTok{  }\KeywordTok{mutate}\NormalTok{(}\DataTypeTok{combinedTotal =} \KeywordTok{sum}\NormalTok{(totalAmount), }\DataTypeTok{totalRelative =}\NormalTok{ totalAmount }\OperatorTok{/}\StringTok{ }\NormalTok{(combinedTotal)) }

\NormalTok{candidateDataEarlyClean <-}\StringTok{ }\NormalTok{candidateDataEarlyClean }\OperatorTok
\StringTok{  }\KeywordTok{group_by}\NormalTok{(district) }\OperatorTok
\StringTok{  }\KeywordTok{mutate}\NormalTok{(}\DataTypeTok{combinedTotal =} \KeywordTok{sum}\NormalTok{(totalAmount), }\DataTypeTok{totalRelative =}\NormalTok{ totalAmount }\OperatorTok{/}\StringTok{ }\NormalTok{(combinedTotal)) }\OperatorTok
\StringTok{  }\KeywordTok{mutate}\NormalTok{(}\DataTypeTok{totalRelative =} \ControlFlowTok{if}\NormalTok{(}\OperatorTok{!}\KeywordTok{is.finite}\NormalTok{(totalRelative)[}\DecValTok{1}\NormalTok{]) }\FloatTok{0.5} \ControlFlowTok{else}\NormalTok{ totalRelative)}
\end{Highlighting}
\end{Shaded}

\hypertarget{plot-early-self-funding-against-total-amount-raised}{%
\subsubsection{Plot early self funding against total amount
raised}\label{plot-early-self-funding-against-total-amount-raised}}

\begin{Shaded}
\begin{Highlighting}[]
\NormalTok{candidateDataEarlyClean }\OperatorTok
\StringTok{  }\KeywordTok{filter}\NormalTok{(win }\OperatorTok{==}\StringTok{ }\DecValTok{1}\NormalTok{) }\OperatorTok
\StringTok{  }\KeywordTok{ggplot}\NormalTok{(}\DataTypeTok{mapping =} \KeywordTok{aes}\NormalTok{(}\DataTypeTok{x =}\NormalTok{ earlySelfFundedRelative, }\DataTypeTok{y =}\NormalTok{ totalRelative, }\DataTypeTok{size =}\NormalTok{ earlySelfFundedAmt)) }\OperatorTok{+}\StringTok{ }
\StringTok{  }\KeywordTok{geom_point}\NormalTok{() }\OperatorTok{+}\StringTok{ }
\StringTok{  }\KeywordTok{ggtitle}\NormalTok{(}\StringTok{"Early Self Funding vs Total Amount Raised"}\NormalTok{) }\OperatorTok{+}
\StringTok{  }\KeywordTok{xlab}\NormalTok{(}\StringTok{"Relative Early Self Funding"}\NormalTok{) }\OperatorTok{+}\StringTok{ }
\StringTok{  }\KeywordTok{ylab}\NormalTok{(}\StringTok{"Relative Total Raised"}\NormalTok{) }\OperatorTok{+}
\StringTok{  }\KeywordTok{labs}\NormalTok{(}
    \DataTypeTok{size =} \StringTok{"Early Self Funding ($)"}
\NormalTok{  ) }\OperatorTok{+}
\StringTok{  }\KeywordTok{geom_smooth}\NormalTok{()}
\end{Highlighting}
\end{Shaded}

\begin{verbatim}
## `geom_smooth()` using method = 'loess' and formula 'y ~ x'
\end{verbatim}

\includegraphics{final-story_files/figure-latex/unnamed-chunk-15-1.pdf}

\begin{Shaded}
\begin{Highlighting}[]
\NormalTok{candidateDataEarlyNaN }\OperatorTok
\StringTok{  }\KeywordTok{filter}\NormalTok{(win }\OperatorTok{==}\StringTok{ }\DecValTok{1}\NormalTok{, }\KeywordTok{is.finite}\NormalTok{(earlySelfFundedRelative)[}\DecValTok{1}\NormalTok{]) }\OperatorTok
\StringTok{  }\KeywordTok{ggplot}\NormalTok{(}\DataTypeTok{mapping =} \KeywordTok{aes}\NormalTok{(}\DataTypeTok{x =}\NormalTok{ earlySelfFundedRelative, }\DataTypeTok{y =}\NormalTok{ totalRelative, }\DataTypeTok{size =}\NormalTok{ earlySelfFundedAmt)) }\OperatorTok{+}\StringTok{ }
\StringTok{  }\KeywordTok{geom_point}\NormalTok{() }\OperatorTok{+}\StringTok{ }
\StringTok{  }\KeywordTok{ggtitle}\NormalTok{(}\StringTok{"Early Self Funding vs Total Amount Raised"}\NormalTok{) }\OperatorTok{+}
\StringTok{  }\KeywordTok{xlab}\NormalTok{(}\StringTok{"Relative Early Self Funding"}\NormalTok{) }\OperatorTok{+}\StringTok{ }
\StringTok{  }\KeywordTok{ylab}\NormalTok{(}\StringTok{"Relative Total Raised"}\NormalTok{) }\OperatorTok{+}
\StringTok{  }\KeywordTok{labs}\NormalTok{(}
    \DataTypeTok{size =} \StringTok{"Early Self Funding ($)"}
\NormalTok{  ) }\OperatorTok{+}
\StringTok{  }\KeywordTok{geom_smooth}\NormalTok{()}
\end{Highlighting}
\end{Shaded}

\begin{verbatim}
## `geom_smooth()` using method = 'loess' and formula 'y ~ x'
\end{verbatim}

\includegraphics{final-story_files/figure-latex/unnamed-chunk-15-2.pdf}

\hypertarget{compare-early-self-funding-to-margin-of-victory}{%
\subsubsection{Compare early self funding to margin of
victory}\label{compare-early-self-funding-to-margin-of-victory}}

\begin{Shaded}
\begin{Highlighting}[]
\NormalTok{candidateDataEarlyClean }\OperatorTok
\StringTok{  }\CommentTok{#filter(earlySelfFundedAmt > 0) %>%}
\StringTok{  }\KeywordTok{filter}\NormalTok{(win }\OperatorTok{==}\StringTok{ }\DecValTok{1}\NormalTok{) }\OperatorTok
\StringTok{  }\KeywordTok{ggplot}\NormalTok{(}\DataTypeTok{mapping =} \KeywordTok{aes}\NormalTok{(}\DataTypeTok{x =}\NormalTok{ earlySelfFundedRelative, }\DataTypeTok{y =}\NormalTok{ margin)) }\OperatorTok{+}\StringTok{ }
\StringTok{  }\KeywordTok{geom_point}\NormalTok{(}\DataTypeTok{size=}\DecValTok{3}\NormalTok{) }\OperatorTok{+}\StringTok{ }
\StringTok{  }\KeywordTok{ggtitle}\NormalTok{(}\StringTok{"Early Self Funding vs Margin of Victory"}\NormalTok{) }\OperatorTok{+}
\StringTok{  }\KeywordTok{xlab}\NormalTok{(}\StringTok{"Relative Early Self Funding"}\NormalTok{) }\OperatorTok{+}\StringTok{ }
\StringTok{  }\KeywordTok{ylab}\NormalTok{(}\StringTok{"Margin of Victory (percentage)"}\NormalTok{) }\OperatorTok{+}
\StringTok{  }\KeywordTok{geom_smooth}\NormalTok{()}
\end{Highlighting}
\end{Shaded}

\begin{verbatim}
## `geom_smooth()` using method = 'loess' and formula 'y ~ x'
\end{verbatim}

\includegraphics{final-story_files/figure-latex/unnamed-chunk-16-1.pdf}

\hypertarget{calculate-close-funding-amounts-before-the-primary}{%
\subsubsection{Calculate close funding amounts before the
primary}\label{calculate-close-funding-amounts-before-the-primary}}

\begin{Shaded}
\begin{Highlighting}[]
\NormalTok{candidateDataEarlyClose <-}\StringTok{ }\NormalTok{candidateData}

\CommentTok{# Given a dataframe of receipts, a last name, and a first name/aliases, calculates the amount of early self funding a candidate received}
\CommentTok{# Returns the total amount received through early close funding}
\NormalTok{calculateEarlyCloseFunding <-}\StringTok{ }\ControlFlowTok{function}\NormalTok{(receipts, lastname, firstnames) \{}
  \CommentTok{# Regex pattern that matches any instance where the candidate's first name/aliases precedes their last name, with anything in between and at the beginning and end of strings}
\NormalTok{  pattern <-}\StringTok{ }\KeywordTok{sprintf}\NormalTok{(}\StringTok{".*%s.*"}\NormalTok{, lastname)}
  
  \CommentTok{# Filters by using grepping with the above pattern on the donors}
\NormalTok{  receipts <-}\StringTok{ }\NormalTok{receipts[ }\KeywordTok{grepl}\NormalTok{(pattern, receipts}\OperatorTok{$}\NormalTok{receipts2018.donor, }\DataTypeTok{ignore.case=}\OtherTok{TRUE}\NormalTok{) }\OperatorTok{|}\StringTok{ }\KeywordTok{grepl}\NormalTok{(pattern, receipts}\OperatorTok{$}\NormalTok{receipts2018.employer, }\DataTypeTok{ignore.case=}\OtherTok{TRUE}\NormalTok{), ]}
\NormalTok{  receipts <-}\StringTok{ }\NormalTok{receipts }\OperatorTok
\StringTok{    }\KeywordTok{filter}\NormalTok{(}\KeywordTok{ymd}\NormalTok{(receipts2018.date) }\OperatorTok{<}\StringTok{ }\KeywordTok{mdy}\NormalTok{(}\StringTok{"5/8/2018"}\NormalTok{))}
  
  \CommentTok{# Uncomment this print statement to check for discrepancy in resulting list of early close-funded receipts}
  \CommentTok{# print(receipts)}
  
\NormalTok{  total <-}\StringTok{ }\KeywordTok{sum}\NormalTok{(receipts}\OperatorTok{$}\NormalTok{receipts2018.amount)}
  \KeywordTok{return}\NormalTok{(total)}
\NormalTok{\}}

\NormalTok{earlyCloseFundedAll <-}\StringTok{ }\KeywordTok{c}\NormalTok{()}

\CommentTok{# For each candidate, calculate their early close-funded total}
\ControlFlowTok{for}\NormalTok{ (row }\ControlFlowTok{in} \DecValTok{1}\OperatorTok{:}\KeywordTok{nrow}\NormalTok{(candidateData)) \{}
\NormalTok{    district <-}\StringTok{ }\NormalTok{candidateData[row, }\StringTok{"district"}\NormalTok{]}
\NormalTok{    lastname <-}\StringTok{ }\NormalTok{candidateData[row, }\StringTok{"lastname"}\NormalTok{] }\OperatorTok\StringTok{ }\KeywordTok{tolower}\NormalTok{()}
\NormalTok{    firstnames <-}\StringTok{ }\KeywordTok{c}\NormalTok{(candidateData[row, }\StringTok{"firstname"}\NormalTok{] }\OperatorTok\StringTok{ }\KeywordTok{tolower}\NormalTok{())}
    \CommentTok{# <- append(strsplit(candidateData[row, "aliases"], "/"), candidateData[row, "firstname"])}
\NormalTok{    receipts <-}\StringTok{ }\NormalTok{candidateData[row, }\StringTok{"receipts2018"}\NormalTok{]}
\NormalTok{    receipts <-}\StringTok{ }\KeywordTok{do.call}\NormalTok{(data.frame, receipts)}
    
\NormalTok{    earlyCloseFundedAmt <-}\StringTok{ }\KeywordTok{calculateEarlyCloseFunding}\NormalTok{(receipts, lastname, firstnames)}

\NormalTok{    earlyCloseFundedAll <-}\StringTok{ }\KeywordTok{append}\NormalTok{(earlyCloseFundedAll, earlyCloseFundedAmt)}
\NormalTok{\}}

\CommentTok{# Update the candidate data with their close-funded totals}
\NormalTok{candidateDataEarlyClose}\OperatorTok{$}\NormalTok{earlyCloseFundedAmt <-}\StringTok{ }\NormalTok{earlyCloseFundedAll}
\end{Highlighting}
\end{Shaded}

\hypertarget{calculate-relative-early-self-funding-compared-to-opponent-1}{%
\subsubsection{Calculate relative early self funding compared to
opponent}\label{calculate-relative-early-self-funding-compared-to-opponent-1}}

\hypertarget{for-candidate-well-get-the-relative-funding-compared-to-their-district-opponent-in-the-other-party-2}{%
\paragraph{For candidate, we'll get the relative funding compared to
their district opponent in the other
party}\label{for-candidate-well-get-the-relative-funding-compared-to-their-district-opponent-in-the-other-party-2}}

\begin{Shaded}
\begin{Highlighting}[]
\NormalTok{candidateDataEarlyCloseNaN <-}\StringTok{ }\NormalTok{candidateDataEarlyClose }\OperatorTok
\StringTok{  }\KeywordTok{group_by}\NormalTok{(district) }\OperatorTok
\StringTok{  }\KeywordTok{mutate}\NormalTok{(}\DataTypeTok{earlyCloseFundedTotal =} \KeywordTok{sum}\NormalTok{(earlyCloseFundedAmt), }\DataTypeTok{earlyCloseFundedRelative =}\NormalTok{ earlyCloseFundedAmt }\OperatorTok{/}\StringTok{ }\NormalTok{(earlyCloseFundedTotal)) }

\NormalTok{candidateDataEarlyCloseClean <-}\StringTok{ }\NormalTok{candidateDataEarlyCloseNaN }\OperatorTok
\StringTok{  }\KeywordTok{mutate}\NormalTok{(}\DataTypeTok{earlyCloseFundedRelative =} \ControlFlowTok{if}\NormalTok{(}\OperatorTok{!}\KeywordTok{is.finite}\NormalTok{(earlyCloseFundedRelative)[}\DecValTok{1}\NormalTok{]) }\FloatTok{0.5} \ControlFlowTok{else}\NormalTok{ earlyCloseFundedRelative)}

\ControlFlowTok{for}\NormalTok{ (row }\ControlFlowTok{in} \DecValTok{1}\OperatorTok{:}\KeywordTok{nrow}\NormalTok{(candidateDataEarlyCloseNaN)) \{}
\NormalTok{  receipts <-}\StringTok{ }\NormalTok{candidateDataEarlyCloseNaN[row, }\StringTok{"receipts2018"}\NormalTok{]}
\NormalTok{  receipts <-}\StringTok{ }\KeywordTok{do.call}\NormalTok{(data.frame, receipts)}
\NormalTok{  candidateDataEarlyCloseNaN[row, }\StringTok{"totalAmount"}\NormalTok{] =}\StringTok{ }\KeywordTok{sum}\NormalTok{(receipts}\OperatorTok{$}\NormalTok{receipts2018.amount)}
\NormalTok{\}}

\ControlFlowTok{for}\NormalTok{ (row }\ControlFlowTok{in} \DecValTok{1}\OperatorTok{:}\KeywordTok{nrow}\NormalTok{(candidateDataEarlyCloseClean)) \{}
\NormalTok{  receipts <-}\StringTok{ }\NormalTok{candidateDataEarlyCloseClean[row, }\StringTok{"receipts2018"}\NormalTok{]}
\NormalTok{  receipts <-}\StringTok{ }\KeywordTok{do.call}\NormalTok{(data.frame, receipts)}
\NormalTok{  candidateDataEarlyCloseClean[row, }\StringTok{"totalAmount"}\NormalTok{] =}\StringTok{ }\KeywordTok{sum}\NormalTok{(receipts}\OperatorTok{$}\NormalTok{receipts2018.amount)}
\NormalTok{\}}

\NormalTok{candidateDataEarlyCloseNaN <-}\StringTok{ }\NormalTok{candidateDataEarlyCloseNaN }\OperatorTok
\StringTok{  }\KeywordTok{group_by}\NormalTok{(district) }\OperatorTok
\StringTok{  }\KeywordTok{mutate}\NormalTok{(}\DataTypeTok{combinedTotal =} \KeywordTok{sum}\NormalTok{(totalAmount), }\DataTypeTok{totalRelative =}\NormalTok{ totalAmount }\OperatorTok{/}\StringTok{ }\NormalTok{(combinedTotal)) }

\NormalTok{candidateDataEarlyCloseClean <-}\StringTok{ }\NormalTok{candidateDataEarlyCloseClean }\OperatorTok
\StringTok{  }\KeywordTok{group_by}\NormalTok{(district) }\OperatorTok
\StringTok{  }\KeywordTok{mutate}\NormalTok{(}\DataTypeTok{combinedTotal =} \KeywordTok{sum}\NormalTok{(totalAmount), }\DataTypeTok{totalRelative =}\NormalTok{ totalAmount }\OperatorTok{/}\StringTok{ }\NormalTok{(combinedTotal)) }\OperatorTok
\StringTok{  }\KeywordTok{mutate}\NormalTok{(}\DataTypeTok{totalRelative =} \ControlFlowTok{if}\NormalTok{(}\OperatorTok{!}\KeywordTok{is.finite}\NormalTok{(totalRelative)[}\DecValTok{1}\NormalTok{]) }\FloatTok{0.5} \ControlFlowTok{else}\NormalTok{ totalRelative)}
\end{Highlighting}
\end{Shaded}

\hypertarget{plot-early-close-funding-against-total-amount-raised}{%
\subsubsection{Plot early close funding against total amount
raised}\label{plot-early-close-funding-against-total-amount-raised}}

\begin{Shaded}
\begin{Highlighting}[]
\NormalTok{candidateDataEarlyCloseClean }\OperatorTok
\StringTok{  }\KeywordTok{filter}\NormalTok{(win }\OperatorTok{==}\StringTok{ }\DecValTok{1}\NormalTok{) }\OperatorTok
\StringTok{  }\KeywordTok{ggplot}\NormalTok{(}\DataTypeTok{mapping =} \KeywordTok{aes}\NormalTok{(}\DataTypeTok{x =}\NormalTok{ earlyCloseFundedRelative, }\DataTypeTok{y =}\NormalTok{ totalRelative, }\DataTypeTok{size =}\NormalTok{ earlyCloseFundedAmt)) }\OperatorTok{+}\StringTok{ }
\StringTok{  }\KeywordTok{geom_point}\NormalTok{() }\OperatorTok{+}\StringTok{ }
\StringTok{  }\KeywordTok{ggtitle}\NormalTok{(}\StringTok{"Early Close Funding vs Total Amount Raised"}\NormalTok{) }\OperatorTok{+}
\StringTok{  }\KeywordTok{xlab}\NormalTok{(}\StringTok{"Relative Early Close Funding"}\NormalTok{) }\OperatorTok{+}\StringTok{ }
\StringTok{  }\KeywordTok{ylab}\NormalTok{(}\StringTok{"Relative Total Raised"}\NormalTok{) }\OperatorTok{+}
\StringTok{  }\KeywordTok{labs}\NormalTok{(}
    \DataTypeTok{size =} \StringTok{"Early Close Funding ($)"}
\NormalTok{  ) }\OperatorTok{+}
\StringTok{  }\KeywordTok{geom_smooth}\NormalTok{()}
\end{Highlighting}
\end{Shaded}

\begin{verbatim}
## `geom_smooth()` using method = 'loess' and formula 'y ~ x'
\end{verbatim}

\includegraphics{final-story_files/figure-latex/unnamed-chunk-19-1.pdf}

\begin{Shaded}
\begin{Highlighting}[]
\NormalTok{candidateDataEarlyCloseNaN }\OperatorTok
\StringTok{  }\KeywordTok{filter}\NormalTok{(win }\OperatorTok{==}\StringTok{ }\DecValTok{1}\NormalTok{, }\KeywordTok{is.finite}\NormalTok{(earlyCloseFundedRelative)[}\DecValTok{1}\NormalTok{]) }\OperatorTok
\StringTok{  }\KeywordTok{ggplot}\NormalTok{(}\DataTypeTok{mapping =} \KeywordTok{aes}\NormalTok{(}\DataTypeTok{x =}\NormalTok{ earlyCloseFundedRelative, }\DataTypeTok{y =}\NormalTok{ totalRelative, }\DataTypeTok{size =}\NormalTok{ earlyCloseFundedAmt)) }\OperatorTok{+}\StringTok{ }
\StringTok{  }\KeywordTok{geom_point}\NormalTok{() }\OperatorTok{+}\StringTok{ }
\StringTok{  }\KeywordTok{ggtitle}\NormalTok{(}\StringTok{"Early Close Funding vs Total Amount Raised"}\NormalTok{) }\OperatorTok{+}
\StringTok{  }\KeywordTok{xlab}\NormalTok{(}\StringTok{"Relative Early Close Funding"}\NormalTok{) }\OperatorTok{+}\StringTok{ }
\StringTok{  }\KeywordTok{ylab}\NormalTok{(}\StringTok{"Relative Total Raised"}\NormalTok{) }\OperatorTok{+}
\StringTok{  }\KeywordTok{labs}\NormalTok{(}
    \DataTypeTok{size =} \StringTok{"Early Close Funding ($)"}
\NormalTok{  ) }\OperatorTok{+}
\StringTok{  }\KeywordTok{geom_smooth}\NormalTok{()}
\end{Highlighting}
\end{Shaded}

\begin{verbatim}
## `geom_smooth()` using method = 'loess' and formula 'y ~ x'
\end{verbatim}

\includegraphics{final-story_files/figure-latex/unnamed-chunk-19-2.pdf}

\hypertarget{compare-early-self-funding-to-margin-of-victory-1}{%
\subsubsection{Compare early self funding to margin of
victory}\label{compare-early-self-funding-to-margin-of-victory-1}}

\begin{Shaded}
\begin{Highlighting}[]
\NormalTok{candidateDataEarlyClean }\OperatorTok
\StringTok{  }\CommentTok{#filter(earlySelfFundedAmt > 0) %>%}
\StringTok{  }\KeywordTok{filter}\NormalTok{(win }\OperatorTok{==}\StringTok{ }\DecValTok{1}\NormalTok{) }\OperatorTok
\StringTok{  }\KeywordTok{ggplot}\NormalTok{(}\DataTypeTok{mapping =} \KeywordTok{aes}\NormalTok{(}\DataTypeTok{x =}\NormalTok{ earlySelfFundedRelative, }\DataTypeTok{y =}\NormalTok{ margin)) }\OperatorTok{+}\StringTok{ }
\StringTok{  }\KeywordTok{geom_point}\NormalTok{(}\DataTypeTok{size=}\DecValTok{3}\NormalTok{) }\OperatorTok{+}\StringTok{ }
\StringTok{  }\KeywordTok{ggtitle}\NormalTok{(}\StringTok{"Early Close Funding vs Margin of Victory"}\NormalTok{) }\OperatorTok{+}
\StringTok{  }\KeywordTok{xlab}\NormalTok{(}\StringTok{"Relative Close Self Funding"}\NormalTok{) }\OperatorTok{+}\StringTok{ }
\StringTok{  }\KeywordTok{ylab}\NormalTok{(}\StringTok{"Margin of Victory (percentage)"}\NormalTok{) }\OperatorTok{+}
\StringTok{  }\KeywordTok{geom_smooth}\NormalTok{()}
\end{Highlighting}
\end{Shaded}

\begin{verbatim}
## `geom_smooth()` using method = 'loess' and formula 'y ~ x'
\end{verbatim}

\includegraphics{final-story_files/figure-latex/unnamed-chunk-20-1.pdf}

\end{document}
